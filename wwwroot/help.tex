\documentclass[12pt, a4paper]{article}
\usepackage{xltxtra}
\usepackage{polyglossia}
\setmainlanguage{russian}
%\setotherlanguage{english}


\setmainfont{Times New Roman}
\setromanfont{Times New Roman} 
\setsansfont{Arial} 
\setmonofont{Courier New} 

\newfontfamily{\cyrillicfont}{Times New Roman} 
\newfontfamily{\cyrillicfontrm}{Times New Roman}
\newfontfamily{\cyrillicfonttt}{Courier New}
\newfontfamily{\cyrillicfontsf}{Arial}


\usepackage{graphicx}
\usepackage{wordlike}
\usepackage{hyperref}
\title{Краткое руководство пользователя}


\begin{document}
\maketitle

\section{Запуск приложения}

При запуске приложения нужно ввести логин и пароль. Логин выдает администратор. 
При первом входе в приложение система попросит задать новый пароль. Для вызова навигации нужно нажать кнопку слева для вызова основного меню. 

\section{Просмотр и редактирование данных} 

Во всех таблицах приложения выводятся первые 30 записей. По двойному клику по строке
появляется окно просмотра выбранной записи.
Для добавления/редактирования/удаления и других действий предназначено контекстное меню, вызывается
по кнопке справа с тремя точками. 


Следующие пункты в контекстном меню есть всегда:

\begin{itemize}
    \item {'Фильтровка и сортировка' - используется для поиска нужной записи и сортировки}
    \item {'Страницы' - используется для показа следующих или предыдущих 30-ти записей }
    \item {'Обновить' - обновляет текущую страницу}
    \item {'Экспорт в CSV' - выгрузка данных в файл CSV}
\end{itemize}

При редактировании данных у некоторых полей справа есть "иконка" с лупой. Это значит, что это поле 
заполняется из другого справочника. Нужно кликнуть по лупе, появится список для выбора значения, в этом списке
нужно выбрать необходимую запись.

\subsection{Фильтровка и сортировка}
По выбору пункта меню: "Фильтровка и сортировка" 
появляется окно для ввода фильтров и порядка сортировки.
При поиске/фильтровке показываются записи у которых в указанном поле есть заданная подстрока.


\section{Пользователи и роли}
Для заведения новых пользователей нужно зайти в систему с правами администратора и выбрать пункт: "Настройки/Пользователи НСИ".
При заведении нового пользователя (логина) или при сохранении существующего у пользователя сбрасывается пароль. 
При входе в систему с новым логином нужно будет задать пароль.

В приложении предустановленны две роли "Администраторы" и "Операторы". 
Все логины нужно отнести к одной из групп: "Администраторы" или "Операторы". 
Для этого нужно открыть форму "Группы пользователей", эта форма доступна только администраторам. 
Выбираем одну из групп и в меню выбираем пункт "Детали" или делаем двойной клик по записи. 
Появится список пользователей, которые относятся к выбранной группе. 
Этот список можно редактировать, добавлять или удалять пользователей из списка (группы)



\section{Коннекторы и запросы} 
Формы коннекторов и запросов к сервису НСИ доступны только администратору. 
В форме запросов доступен пункт меню: "Детали". По этому пункту меню показывается форма 
с результатами запроса. В этой форме результаты запроса можно сохранить в CSV.


\end{document}